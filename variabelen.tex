\documentclass[10pt,a4paper]{article}
\usepackage[utf8]{inputenc}
\usepackage{amsmath}
\usepackage{amsfonts}
\usepackage{amssymb}
\usepackage[left=1.5cm,right=1.5cm,top=1.5cm,bottom=1.5cm]{geometry}

\title{Variabelen in optimalisering en complexiteit}
\author{}
\date{}
\begin{document}
\section*{Variabelen in optimalisering en complexiteit}

Aan dit document kunnen geen rechten worden ontleend. Er is geen garantie van correctheid van dit document. Dit document zal regelmatig geüpdate worden met verbeteringen en nieuwe variabelen. Houd het dus in de gaten.


\begin{tabular}{|c|l|}
\hline
$\textbf{M}_{ij}$ & Enkel element van matrix $M$ uit \textit{rij} $i$ en \textit{kolom} $j$. $\textbf{M}_{11}$ is het element linksboven.\\
\hline
$\textbf{m}_i$ & Kolom i uit matrix $\textbf{M}$. Let op: kleine, dikgedrukte letter met kolomnummer als subscript. \\
\hline
$\textbf{m}^i$ & Rij i uit matrix $\textbf{M}$. \\
\hline 
$\textbf{c}$ & Kostenvector. Bevat alle kostencoëfficiënten.\\ & Deze kosten zijn altijd constant.\\ 

 & $c_1, c_1, ... c_n$ zijn de individuele kosten. \\ & $\textbf{c} = \{c_1, c_2, ..., c_n\}$. \\ 
\hline 
$\textbf{x}$ & Vector van beslissingsvariabelen. \\ & Hiervan worden de waarden veranderd door de simplex methode. \\ & $\textbf{x} = \{x_1, x_2, ... x_n\}$. Let op: een enkele beslissingsvariable $x_i$ is \textit{niet} dikgedrukt! \\
\hline
$\textbf{x}^*$ & De invulling van waarden voor $\textbf{x}$ die de optimale oplossing karakteriseert. \\ % Zie 1.3
\hline
$\textbf{x}_j$ & In het boek (hoofdstuk 2 en 3) wordt de \textit{dikgedrukte} $\textbf{x}_j$ gebruikt om de extreme punten \\ 
 & van het toegelaten gebied aan te duiden. Dit is erg verwarrend! \\
\hline
$\textbf{A}$ & Constraint matrix. Constant. \\ 
 & Rijen hieruit zijn vectors van constanten die de lineaire functie in de constraint vormen. \\
 & $\textbf{A}$ is op te delen in matrices $\textbf{B}$ en $\textbf{N}$ (geschreven: $\textbf{A} = [\textbf{B}, \textbf{N}]$.) \\ % Zie 2.3
\hline
$a_{11}, a_{12}, ..., a_{mn}$ & Naam uit boek: technological coefficients. Elementen uit de constraint matrix $\textbf{A}$.\\
& Iedere $a_{i1}, a_{i2}, ... a_{in}$ wordt vermenigvuldigd met $x_1, x_2, ... x_n$ in de constraints. \\
\hline
$\textbf{B}$ & Basismatrix. Deze is altijd regulier (vierkant) en inverteerbaar. \\ % Zie 2.3
& $\textbf{B}$ bestaat uit een selectie van kolommen uit $\textbf{A}$.  \\
\hline
$\textbf{N}$ & Non-basismatrix. $\textbf{N}$ bestaat uit de kolommen van $\textbf{A}$ die niet in de basis $\textbf{B}$ zitten. \\
\hline
$\textbf{x}_B$ & De beslissingsvariabelen die overeenkomen met de basismatrix. \\
& Stel $\textbf{B} = [\textbf{a}_3, \textbf{a}_5, \textbf{a}_2]$ (kolommen 3, 5 en 2 uit matrix $\textbf{A}$) 
 Dan is $\textbf{x}_B$ = $\{x_3, x_5, x_2\}$\\
\hline
$\textbf{b}$ & Right-hand-side vector. Bevat de waarden waar de constraints aan moeten voldoen.\\
 & $\textbf{b} = \{b_1, b_2, ..., b_n\}$\\
\hline
$\textbf{cx}$ & Objective function/criterion function/doelfunctie. \\ & 	Dit is wat je maximaliseert of minimaliseert.\\
\hline
$\displaystyle\sum\limits_{j=1}^n c_jx_j$ & Objective function. Andere schrijfwijze. \\
\hline
$\textbf{z}$ & Objective value. De uitkomstwaarde van de objective function. $\textbf{z} = \textbf{cx}$.\\
\hline
$\textbf{z}^*$ & De objective value van de optimale oplossing, bijv. de maximale winst in euro's. \\ % Zie 1.3
 & $\textbf{z}^* = \textbf{cx}^*$ \\
\hline
$\textbf{Ax} \geq \textbf{b}$ & Alle constraints (behalve de non-negativity constraints.)\\
$\textbf{Ax} \leq \textbf{b}$ & Let op, dit is alleen in de matrixnotatie te schrijven als \\
$\textbf{Ax} = \textbf{b}$ & het (on)gelijkheidsteken hetzelfde is bij elke constraint.\\
\hline
$\displaystyle\sum\limits_{j=1}^n a_{ij}x_j \ge b_i$ & Één constraint, oftewel één rij uit de ``subject to''.\\
 & Hier is te zien dat dat er een rij uit $\textbf{A}$ wordt gepakt, vermenigvuldigd \\ &  wordt met $\textbf{x}$ en vergeleken wordt met een element uit $\textbf{b}$. \\
\hline
$x_1, x_2, ..., x_n \geq 0$ & Nonnegativity constraints. \\ & Triviale constraints dat je beslissingsvariabelen niet negatief mogen worden.\\
\hline
\end{tabular}

\newpage
\section*{Termen}
\begin{tabular}{|c|l|}
\hline
Feasible point & Een assignment van waarden voor variabelen $x_1, ..., x_n$ die voldoet aan alle constraints. \\ 
& Een feasible point is een mogelijke oplossing voor het probleem.\\
\hline
Feasible region & De ruimte die bestaat uit alle feasible points. Heeft de vorm van een simplex. \\
\hline
Slack variable & Beslissingsvariabele die toegevoegd wordt om een constraint in de vorm \\ &van een ongelijkheid naar een gelijkheid (van b.v. $\textbf{Ax} \geq \textbf{b}$ naar $\textbf{Ax} = \textbf{b}$.)\\
& Bij een slack variable hoort weer een extra kolom bij de $\textbf{A}$ matrix. \\
\hline
Standard form & Notatie van optimaliseringsprobleem waar alle constraints gelijkheden zijn.\\
\hline
Canonical form & Notatie van optimaliseringsprobleem waar alle constraints ongelijkheden zijn.\\
 & Specifiek: $\le$ bij maximalisering en $\geq$ bij minimalisering. \\
%\hline
% & \\
\hline
\end{tabular} 
\end{document}
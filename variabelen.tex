\documentclass[10pt,a4paper]{article}
\usepackage[utf8]{inputenc}
\usepackage{amsmath}
\usepackage{amsfonts}
\usepackage{amssymb}
\usepackage[left=1.5cm,right=1.5cm,top=1.5cm,bottom=1.5cm]{geometry}

\title{Variabelen in optimalisering en complexiteit}
\author{}
\date{}
\begin{document}
\section*{Variabelen in optimalisering en complexiteit}

Aan dit document kunnen geen rechten worden ontleend. Er is geen garantie van correctheid van dit document. Dit document zal regelmatig geüpdate worden met verbeteringen en nieuwe variabelen. Houd het dus in de gaten.


\begin{tabular}{|c|l|}
\hline
$\textbf{M}_{ij}$ & Enkel element van matrix $M$ uit \textit{rij} $i$ en \textit{kolom} $j$. $\textbf{M}_{11}$ is het element linksboven.\\
\hline
$\textbf{m}_i$ & Kolom i uit matrix $\textbf{M}$. Let op: kleine, dikgedrukte letter met kolomnummer als subscript. \\
\hline
$\textbf{m}^i$ & Rij i uit matrix $\textbf{M}$. \\
\hline
$X$ & Hoofdletter, niet dikgedrukt. De set van punten in het toegelaten gebied. \\
\hline
$\textbf{c}$ & Kostenvector. Bevat alle kostencoëfficiënten.\\ & Deze kosten zijn altijd constant.\\ 

 & $c_1, c_1, ... c_n$ zijn de individuele kosten. \\ & $\textbf{c} = \{c_1, c_2, ..., c_n\}$. \\ 
\hline 
$\textbf{x}$ & Vector van beslissingsvariabelen. \\ & Hiervan worden de waarden veranderd door de simplex methode. \\ & $\textbf{x} = \{x_1, x_2, ... x_n\}$. Let op: een enkele beslissingsvariable $x_i$ is \textit{niet} dikgedrukt! \\
\hline
$\textbf{x}^*$ & De invulling van waarden voor $\textbf{x}$ die de optimale oplossing karakteriseert. \\ % Zie 1.3
\hline
$\textbf{x}_j$ & In het boek (hoofdstuk 2 en 3) wordt de \textit{dikgedrukte} $\textbf{x}_j$ gebruikt om de extreme punten \\ 
 & van het toegelaten gebied aan te duiden. Dit is erg verwarrend! \\
 & In de hoorcolleges wordt $\textbf{\^x}$ gebruikt. \\
\hline
$\textbf{A}$ & Constraint matrix. Constant. \\ 
 & Rijen hieruit zijn vectors van constanten die de lineaire functie in de constraint vormen. \\
 & $\textbf{A}$ is op te delen in matrices $\textbf{B}$ en $\textbf{N}$ (geschreven: $\textbf{A} = [\textbf{B}, \textbf{N}]$.) \\ % Zie 2.3
\hline
$a_{11}, a_{12}, ..., a_{mn}$ & Naam uit boek: technological coefficients. Elementen uit de constraint matrix $\textbf{A}$.\\
& Iedere $a_{i1}, a_{i2}, ... a_{in}$ wordt vermenigvuldigd met $x_1, x_2, ... x_n$ in de constraints. \\
\hline
$\textbf{B}$ & Basismatrix. Deze is altijd regulier (vierkant) en inverteerbaar. \\ % Zie 2.3
& $\textbf{B}$ bestaat uit de selectie van kolommen uit $\textbf{A}$ die overeenkomen met de basisvariabelen.  \\
\hline
$\textbf{N}$ & Non-basismatrix. $\textbf{N}$ bestaat uit de kolommen van $\textbf{A}$ die niet in de basis zitten. \\
\hline
$\textbf{x}_B$ & Basisvariabelen. De vector van beslissingsvariabelen die overeenkomen met de basismatrix. \\
& Stel $\textbf{B} = [\textbf{a}_3, \textbf{a}_5, \textbf{a}_2]$ (kolommen 3, 5 en 2 uit matrix $\textbf{A}$) dan is $\textbf{x}_B$ = $\{x_3, x_5, x_2\}$. \\
\hline
$\textbf{x}_N$ & Non-basisvariabelen. De vector van beslissingsvariabelen die niet in de basis zitten. \\
 & $\textbf{x} = \textbf{x}_B \cup \textbf{x}_N$. \\
\hline
$\textbf{c}_B$ en $\textbf{c}_N$ & Hetzelfde idee als $\textbf{x}_B$ en $\textbf{x}_N$, maar dan voor de kostencoëfficienten. \\
\hline
$\textbf{b}$ & Right-hand-side vector. Bevat de waarden waar de constraints aan moeten voldoen.\\
 & $\textbf{b} = \{b_1, b_2, ..., b_n\}$\\
\hline
$\overline{\textbf{b}}$ & $\overline{\textbf{b}} = \textbf{B}^{-1}\textbf{b}$. Omdat $\textbf{B}^{-1}\textbf{b}$ vaak gebruikt wordt, wordt het afgekort met $\overline{\textbf{b}}$. Zie formules. \\
\hline
$J$ & De set van indices van de kolommen uit $\textbf{A}$ die $\textbf{N}$ definiëren (dus de nummers van alle \\
& kolommen die niet in de basis zitten) \\
\hline
$\textbf{y}_j$ & $\textbf{y}_j = \textbf{B}^{-1}\textbf{a}_j$ waarbij $j \in J$. Dit is de vector van waarden die van $\textbf{x}_B$ worden afgetrokken \\ & als je non-basisvariabele $x_j$ met één verhoogt. \\
\hline
$\textbf{cx}$ & Objective function/criterion function/doelfunctie. \\ & 	Dit is wat je maximaliseert of minimaliseert.\\
\hline
$\displaystyle\sum\limits_{j=1}^n c_jx_j$ & Objective function. Andere schrijfwijze. \\
\hline
$z$ & Objective value. De uitkomstwaarde van de objective function. $z = \textbf{cx}$.\\
\hline
$z_0$ & De objective value bij een basisoplossing. $z_0 = \textbf{c}_B\textbf{B}^{-1}\textbf{b}$. Zie formules. \\
\hline
$z_j$ & $z_j = \textbf{c}_B\textbf{B}^{-1}\textbf{a}_j$ voor $j \in J$. Korte schrijfwijze voor een vaak gebruikte formule. \\
\hline
$z^*$ & De objective value van de optimale oplossing, bijv. de maximale winst in euro's. \\ % Zie 1.3
 & $z^* = \textbf{cx}^*$ \\
\hline
$\textbf{Ax} \geq \textbf{b}$ & Alle constraints (behalve de non-negativity constraints.)\\
$\textbf{Ax} \leq \textbf{b}$ & Let op, dit is alleen in de matrixnotatie te schrijven als \\
$\textbf{Ax} = \textbf{b}$ & het (on)gelijkheidsteken hetzelfde is bij elke constraint.\\
\hline
$\displaystyle\sum\limits_{j=1}^n a_{ij}x_j \ge b_i$ & Één constraint, oftewel één rij uit de ``subject to''.\\
 & Hier is te zien dat dat er een rij uit $\textbf{A}$ wordt gepakt, vermenigvuldigd \\ &  wordt met $\textbf{x}$ en vergeleken wordt met een element uit $\textbf{b}$. \\
\hline
$x_1, x_2, ..., x_n \geq 0$ & Nonnegativity constraints. \\ & Triviale constraints dat je beslissingsvariabelen niet negatief mogen worden.\\
\hline
\end{tabular}

\newpage
\section*{Termen}
\begin{tabular}{|c|l|}
\hline
Feasible point & Een assignment van waarden voor variabelen $x_1, ..., x_n$ die voldoet aan alle constraints. \\ 
& Een feasible point is een mogelijke oplossing voor het probleem.\\
\hline
Feasible region & De ruimte die bestaat uit alle feasible points. Heeft de vorm van een simplex. \\
\hline
Slack variable & Beslissingsvariabele die toegevoegd wordt om een constraint in de vorm \\ &van een ongelijkheid naar een gelijkheid (van b.v. $\textbf{Ax} \geq \textbf{b}$ naar $\textbf{Ax} = \textbf{b}$.)\\
& Bij een slack variable hoort weer een extra kolom bij de $\textbf{A}$ matrix. \\
\hline
Standard form & Notatie van optimaliseringsprobleem waar alle constraints gelijkheden zijn.\\
\hline
Canonical form & Notatie van optimaliseringsprobleem waar alle constraints ongelijkheden zijn.\\
 & Specifiek: $\le$ bij maximalisering en $\geq$ bij minimalisering. \\
\hline
Basisoplossing & Een invulling van $\textbf{x}$ die hoort bij een bepaalde basis $\textbf{B}$. \\
 & Bij een basisoplossing geldt altijd dat $\textbf{x}_N = 0$ en $\textbf{x}_B = B^{-1}b$ (zie formules) \\
 & Een basisoplossing is altijd een toegestane oplossing, maar het is \textit{niet} altijd optimaal. \\
 & Een basisoplossing komt overeen met een extreem punt. \\
 & een ''\textit{degenerate basic solution}`` is een basisoplossing waarbij $x_{Bi}$ = 0 voor één of meer $i$. \\
%\hline
% & \\
\hline
\end{tabular} 

\section*{Formules}
\begin{description}
\item[$\textbf{x}_B = \textbf{B}^{-1}\textbf{b}$] \hfill \\
Dit is een basisoplossing (zie termen en hoofdstuk 3.2) in het LP probleem waar geldt $\textbf{Ax} = \textbf{b}$.
Deze basisoplossing hoort bij een basis $\textbf{B}$.

\textbf{Afleiding:} \\
Als matrix $\textbf{A}$ is opgedeeld in matrices $\textbf{B}$ en $\textbf{N}$, dan zijn de constraints ook zo op te delen:

$\textbf{Ax} = \textbf{b}$ bestaat dus uit twee delen: \\
$\textbf{Bx}_B + \textbf{Nx}_N = \textbf{b}$

Met herschrijfregels en de kennis dat $\textbf{x}_N = 0$ (definitie basisoplossing) kunnen we afleiden wat $\textbf{x}_B$ is.

\begin{tabular}{l l l l}
$\textbf{Bx}_B + \textbf{Nx}_N$ & = & $\textbf{b}$ & $\textbf{Ax} = \textbf{b}$ opdelen in basisgedeelte en non-basisgedeelte \\
$\textbf{Bx}_B$ & = & $\textbf{b}$ &  $\textbf{x}_N = 0$, dus die factor kan weggelaten worden \\ 
$\textbf{B}^ {-1}\textbf{Bx}_B$ & = & $\textbf{B}^ {-1}\textbf{b}$ & beide kanten voorvermenigvuldigen met $\textbf{B}^{-1}$ \\
$\textbf{Ix}_B$ & = & $\textbf{B}^ {-1}\textbf{b}$ & omdat geldt: $\textbf{B}^{-1}\textbf{B} = I$ \\
$\textbf{x}_B$ & = & $\textbf{B}^{-1}\textbf{b}$ & \\
\end{tabular}


\item[$z_0 = \textbf{c}_B\textbf{B}^{-1}\textbf{b}$] \hfill \\
Dit is de objectieve function $z = \textbf{cx}$ (zie variabelen), maar dan ingevuld met een basisoplossing (zie termen).

\textbf{Afleiding:} \\
Vanuit de bekende objective function:

\begin{tabular}{l l}
$z = \textbf{cx}$ & objective function \\
$z = \textbf{c}_B\textbf{x}_B + \textbf{c}_N\textbf{x}_N$ & opdelen in basisgedeelte en non-basisgedeelte \\
$z = \textbf{c}_B\textbf{x}_B + 0$ & aangezien $\textbf{x}_N = 0$ (definitie basisoplossing), kan deze factor worden weggelaten. \\
$z = \textbf{c}_B\textbf{B}^{-1}\textbf{b}$ & Invulling $\textbf{x}_B = \textbf{B}^{-1}\textbf{b}$
\end{tabular}



\item[$z = z_0 - \displaystyle\sum\limits_{j \in J} (z_j - c_j)x_j $] \hfill \\
Bij zowel de formule voor $z_0$ en bij de formule voor $\textbf{x}_B$ is de factor $\textbf{x}_N$ weggelaten omdat $\textbf{x}_N = 0$. Echter, als je de simplex methode toepast, ga je van de ene basis naar de andere. Hierbij is het fijn om te weten welke variabelen je in de basis wil hebben. Om hierachter te komen, moet je weten welke variabelen de grootste verbetering oplevert. Hierbij heb je een definitie van de objective value nodig waar $\textbf{x}_N$ wel meegerekend wordt.

\textbf{Afleiding:} \\
Definitie voor $\textbf{x}_B$ zonder de aanname dat $\textbf{x}_N = 0$ (uit het boek, hoofdstuk 3.3):

\begin{tabular}{l l l l}
$\textbf{Bx}_B + \textbf{Nx}_N$ & = & $\textbf{b}$ & $\textbf{Ax} = \textbf{b}$ opdelen in basisgedeelte en non-basisgedeelte \\
$\textbf{B}^ {-1}\textbf{Bx}_B + \textbf{B}^ {-1}\textbf{Nx}_N$ & = & $\textbf{B}^ {-1}\textbf{b}$ & beide kanten voorvermenigvuldigen met $\textbf{B}^{-1}$ \\
$\textbf{Ix}_B + \textbf{B}^ {-1}\textbf{Nx}_N$ & = & $\textbf{B}^ {-1}\textbf{b}$ & omdat geldt: $\textbf{B}^{-1}\textbf{B} = I$ \\
$\textbf{x}_B + \textbf{B}^ {-1}\textbf{Nx}_N$ & = & $\textbf{B}^ {-1}\textbf{b}$ & \\
$\textbf{x}_B$ & = & $\textbf{B}^{-1}\textbf{b} - \textbf{B}^ {-1}\textbf{Nx}_N$ & naar de andere kant halen \\
$\textbf{x}_B$ & = & $\textbf{B}^{-1}\textbf{b} - \displaystyle\sum\limits_{j \in J}\textbf{B}^ {-1}\textbf{a}_j x_j$ & Herschrijven in sommatievorm, zie variabelen voor J \\
\end{tabular}

Nu de definitie voor de objective function $z$, ook zonder de aanname dat $\textbf{x}_N = 0$. Deze komt ook uit het boek, hoofdstuk 3.3, maar in het boek nemen ze erg grote stappen. De stappen hier zijn een stuk kleiner. Dit moet een stuk beter te begrijpen zijn.

\begin{tabular}{l l}
$z = \textbf{cx}$ & formule objective function \\
$z = \textbf{c}_B \textbf{x}_B + \textbf{c}_N \textbf{x}_N$ & opdelen in basisgedeelte en non-basis gedeelte \\
$z = \textbf{c}_B \textbf{x}_B + \displaystyle\sum\limits_{j \in J}c_j x_j$ & herschrijven naar sommatienotatie, zie variabelen voor $J$ \\
$z = \textbf{c}_B (\textbf{B}^{-1}\textbf{b} - \displaystyle\sum\limits_{j \in J}\textbf{B}^ {-1}\textbf{a}_j x_j) + \displaystyle\sum\limits_{j \in J}c_j x_j$ & De definitie van $\textbf{x}_B$ van zojuist invullen voor $\textbf{x}_B$ \\
$z = \textbf{c}_B \textbf{B}^{-1}\textbf{b} - \displaystyle\sum\limits_{j \in J}\textbf{c}_B \textbf{B}^ {-1}\textbf{a}_j x_j + \displaystyle\sum\limits_{j \in J}c_j x_j$ & haakjes wegwerken \\
$z = z_0 - \displaystyle\sum\limits_{j \in J}\textbf{c}_B \textbf{B}^ {-1}\textbf{a}_j x_j + \displaystyle\sum\limits_{j \in J}c_j x_j$ & er geldt $z_0 = \textbf{c}_B\textbf{B}^{-1}\textbf{b}$ (zie formules), herschrijven dus \\
$z = z_0 - \displaystyle\sum\limits_{j \in J}z_j x_j + \displaystyle\sum\limits_{j \in J}c_j x_j$ & er geldt $z_j = \textbf{c}_B \textbf{B}^ {-1}\textbf{a}_j$ (zie variabelen), herschrijven dus \\
$z = z_0 - \displaystyle\sum\limits_{j \in J}(z_j - c_j) x_j$ & sommaties samenvoegen \\
\end{tabular}



\item[$(z_j - c_j)x_j$ (of nog korter: $z_j - c_j$)] \hfill Zie hoofdstuk 3.4 \\
Dit is een stukje uit de formule $z = z_0 - \displaystyle\sum\limits_{j \in J} (z_j - c_j)x_j$ die heel vaak los van zijn formule is te zien. In de formule is de objectieve waarde van alle basisingsvariabelen verwerkt in het enkele getal $z_0$. De objectieve waarde van alle non-basisvariabelen staan in de formule als $\displaystyle\sum\limits_{j \in J} (z_j - c_j)x_j$ uitgedrukt. De uitleg van $(z_j - c_j)x_j$ is het best op te delen in een stuk over $z_j$ en een stukje over $c_j$:

\textbf{Betekenis van $z_j$:}


In de lijst met variabelen (en in de afleiding van de vorige formule) staat dat $z_j$ een afkorting is: \\
$z_j = c_B \textbf{y}_j = c_B(B^{-1}\textbf{a}_j)$. Dit zijn de kostencoëfficiënten van de basisvariabelen ($c_B$), vermenigvuldigd met $B^{-1}$ vermenigvuldigd met de kolom uit $A$ die bij non-basisvariabele $j$ hoort.

Nu is gegeven (hoofdstuk 3.4) dat als je de waarde van een non-basisvariabele $x_k$ verhoogt (terwijl je de andere non-basisvariabelen op 0 houdt), dan heeft dat een effect op de waardes van de basisvariabelen. 
Stel je verhoogt de waarde van non-basisvariabele $x_k$ met het getal $1$. Omdat $x_k$ niet in de basis zat, was de waarde ervan $0$. De nieuwe waarde is dus $1$. Na deze verhoging veranderen alle basisvariabelen $x_{B1}, x_{B2}, ..., x_{Bm}$ respectievelijk met waardes $-1 \cdot y_{1k}, -1 \cdot y_{2k}, ..., -1 \cdot y_{mk}$. In kolomnotatie: $\textbf{x}_B \rightarrow \textbf{x}_B - 1 \cdot \textbf{y}_k$. 

Aangezien de waardes van de basisvariabelen veranderen, verandert ook hun bijdrage aan de objectieve waarde. $z_0 \rightarrow z_0 - 1 \cdot \textbf{c}_B \textbf{y}_k$. Dit staat precies zo in de formule $z = \underline{z_0 -} \displaystyle\sum\limits_{j \in J} (\underline{z_j} - c_j)x_j$, omdat $\textbf{c}_B \textbf{y}_k = z_j$.

\underline{Kortom:} $z_j$ is hoeveel de bijdrage van de basisvariabelen aan de objective value wordt \textit{verminderd} per eenheid dat je non-basisvariabele $x_j$ verhoogt. Letterlijk uit het boek (p. 111):
\begin{quote}
	The saving (a negative saving means more cost) that results from the modification of the basic variables, as the result of increasing $x_k$ by one unit, [...], is $z_k$
\end{quote}

\textbf{Betekenis van $c_j$}

Het gehele effect van het verhogen van non-basisvariabele $x_j$ is $(z_j - c_j) x_j$. Als je $x_j$ verhoogt, veranderen niet alleen de waardes van de basisvariabelen. De waarde van $x_j$ zelf verandert natuurlijk ook! Dit brengt kosten met zich mee, namelijk $c_j$. 

\textbf{Nu samen in $z_j - c_j$}


$z_j - c_j$ is dus hoeveel de winst verandert (vermindert) als je non-basisvariabele $x_j$ met één verhoogt. Als je minimaliseert, wil je dat $z_j - c_j$ positief is. Verhoging van $x_j$ betekent dan immers een verlaging van de objectieve waarde $z$! Bij maximaliseren is het precies andersom. Dan wil je dat $z_j - c_j < 0$. 

\end{description}

\end{document}